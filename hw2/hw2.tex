\documentclass[a4paper,oneside]{article}
\usepackage[T1]{fontenc}
\usepackage[utf8]{inputenc}
\usepackage[english]{babel}

\usepackage[margin=2.54cm]{geometry}
\usepackage{amsmath}
\usepackage{siunitx}
\usepackage{listings}
\usepackage{color}
\usepackage{textcomp}
\usepackage{graphicx}
\usepackage{subcaption}
%\usepackage{changepage}
\usepackage[section]{placeins}
\usepackage{hyperref}

\definecolor{matlabgreen}{RGB}{28,172,0}
\definecolor{matlablilas}{RGB}{170,55,241}

\newcommand{\includecode}[1]{\lstinputlisting[caption={\ttfamily #1.m},label={lst:#1}]{matlab/#1.m}}
\newcommand{\inlinecode}[1]{\lstinline[basicstyle=\ttfamily,keywordstyle={},stringstyle={},commentstyle={\itshape}]{#1}}

\renewcommand{\vec}[1]{\underline{#1}}
\renewcommand{\Re}[1]{\operatorname{Re}\left[#1\right]}
\newcommand{\E}[1]{\operatorname{E}\left[#1\right]}
\newcommand{\norm}[1]{\left\lVert#1\right\rVert}
\newcommand{\abs}[1]{\left|#1\right|}
\newcommand{\F}[1]{\operatorname{\mathcal{F}}\left[#1\right]}
\newcommand{\ceil}[1]{\left\lceil#1\right\rceil}
\newcommand{\Prob}[1]{\operatorname{P}\left[#1\right]}
\newcommand{\ProbC}[2]{\operatorname{P}\left[#1\middle|#2\right]}
%\newcommand{\calU}[0]{\mathcal{U}}
\newcommand{\ind}[1]{\operatorname{\mathbbm{1}}\left\{#1\right\}}

\newcolumntype{M}{>{$}r<{$}}

\author{Riccardo Zanol}
\title{Homework 2}

\begin{document}
\lstset{
  language=Matlab,
  basicstyle={\ttfamily \footnotesize},
  breaklines=true,
  morekeywords={true,false,warning,xlim,ylim},
  keywordstyle=\color{blue},
  stringstyle=\color{matlablilas},
  commentstyle={\color{matlabgreen} \itshape},
  numberstyle={\ttfamily \tiny},
  frame=leftline,
  showstringspaces=false,
  numbers=left,
  upquote=true,
}
\maketitle
\section*{1}
The class \inlinecode{lcg} implements a Linear Congruential Generator,
which is a RNG that generates numbers according to
\[ x_{k} = ax_{k-1} +b \mod m . \]
To replicate Figure~6.5 a LCG with parameters
$a=16807$,~$b=0$~and~$m=2^{31}-1$ is initialized with the seed value
$x_0 = 1$, the first 1000 numbers that it generates are then sorted
and plotted in Fig.~\ref{plot:qqplot}, where they look to be uniformly
distributed in $[0,1]$.  From their autocorrelation
(Fig.~\ref{plot:autocorr}) and their lag plots
(Fig.~\ref{plot:lagplot}) the generated numbers appear to be
uncorrelated.
\begin{figure}[htbp]
  \centering
  \begin{subfigure}{0.5\textwidth}
    \centering
    \includegraphics[width=0.95\textwidth]{matlab/uniform_qq}
    \caption{QQ plot}
    \label{plot:qqplot}
  \end{subfigure}%
  \begin{subfigure}{0.5\textwidth}
        \centering
    \includegraphics[width=0.95\textwidth]{matlab/rand1_autocorr}
    \caption{Autocorrelation}
    \label{plot:autocorr}
  \end{subfigure}
  \begin{subfigure}{0.5\textwidth}
        \centering
    \includegraphics[width=0.95\textwidth]{matlab/rand1_lagplot}
    \caption{Lag plot}
    \label{plot:lagplot}
  \end{subfigure}
  \caption{Replicas of Figure 6.5}
\end{figure}

\section*{2}

\begin{table}[h]
  \centering
  \begin{tabular}{MM|MMM}
    \multicolumn{2}{c}{Binomial parameters} &
    \multicolumn{1}{c}{CDF inversion} &
    \multicolumn{1}{c}{Bernoullian successes} &
    \multicolumn{1}{c}{Geometric sequence of zeros} \\
    \hline
    n = 10 & p = 0.01 & \SI{2.669}{\s} & \SI{0.851}{\s} & \SI{4.060}{\s} \\
    n = 10 & p = 0.50 & \SI{2.653}{\s} & \SI{0.835}{\s} & \SI{15.525}{\s} \\
    n = 10 & p = 0.90 & \SI{2.693}{\s} & \SI{0.820}{\s} & \SI{24.863}{\s} \\
    n = 50 & p = 0.01 & \SI{2.656}{\s} & \SI{0.960}{\s} & \SI{5.059}{\s} \\
    n = 50 & p = 0.50 & \SI{2.693}{\s} & \SI{1.060}{\s} & \SI{60.092}{\s} \\
    n = 50 & p = 0.90 & \SI{2.843}{\s} & \SI{1.006}{\s} & \SI{106.020}{\s} \\
    n = 100 & p = 0.01 & \SI{2.673}{\s} & \SI{1.125}{\s} & \SI{6.223}{\s} \\
    n = 100 & p = 0.50 & \SI{2.795}{\s} & \SI{1.266}{\s} & \SI{119.626}{\s} \\
    n = 100 & p = 0.90 & \SI{2.925}{\s} & \SI{1.159}{\s} & \SI{207.903}{\s} \\
    n = 200 & p = 0.01 & \SI{2.675}{\s} & \SI{1.514}{\s} & \SI{8.465}{\s} \\
    n = 200 & p = 0.50 & \SI{2.945}{\s} & \SI{1.729}{\s} & - \\
    n = 200 & p = 0.90 & \SI{3.158}{\s} & \SI{1.635}{\s} & - \\
    n = 1000 & p = 0.01 & - & \SI{4.513}{\s} & - \\
    n = 1000 & p = 0.50 & - & \SI{5.584}{\s} & - \\
    n = 1000 & p = 0.90 & - & \SI{4.795}{\s} & - \\
  \end{tabular}
  \caption{Timing of three Binomial generation methods}
  \label{tab:binomial}
\end{table}

\end{document}
