\documentclass[a4paper,oneside]{article}
\usepackage[T1]{fontenc}
\usepackage[utf8]{inputenc}
\usepackage[english]{babel}

\usepackage[margin=2.54cm]{geometry}
\usepackage{amsmath}
\usepackage{bbm}
\usepackage{siunitx}
\usepackage{listings}
\usepackage{color}
\usepackage{textcomp}
\usepackage{graphicx}
\usepackage{subcaption}
%\usepackage{changepage}
\usepackage[section]{placeins}
\usepackage{hyperref}

\definecolor{matlabgreen}{RGB}{28,172,0}
\definecolor{matlablilas}{RGB}{170,55,241}

\newcommand{\includecode}[1]{\lstinputlisting[caption={\ttfamily #1.m},label={lst:#1}]{matlab/#1.m}}
\newcommand{\inlinecode}[1]{\lstinline[basicstyle=\ttfamily,keywordstyle={},stringstyle={},commentstyle={\itshape}]{#1}}

\renewcommand{\vec}[1]{\underline{#1}}
\renewcommand{\Re}[1]{\operatorname{Re}\left[#1\right]}
\newcommand{\E}[1]{\operatorname{E}\left[#1\right]}
\newcommand{\norm}[1]{\left\lVert#1\right\rVert}
\newcommand{\abs}[1]{\left|#1\right|}
\newcommand{\F}[1]{\operatorname{\mathcal{F}}\left[#1\right]}
\newcommand{\ceil}[1]{\left\lceil#1\right\rceil}
\newcommand{\floor}[1]{\left\lfloor#1\right\rfloor}
\newcommand{\Prob}[1]{\operatorname{P}\left[#1\right]}
\newcommand{\ProbC}[2]{\operatorname{P}\left[#1\middle|#2\right]}
\newcommand{\ind}[1]{\operatorname{\mathbbm{1}}\left\{#1\right\}}
\newcommand{\distr}[0]{\sim}
\newcommand{\unif}[1]{\mathcal{U}_{#1}}

\newcolumntype{M}{>{$}r<{$}}

\author{Riccardo Zanol}
\title{Homework 3}

\begin{document}
\lstset{
  language=Matlab,
  basicstyle={\ttfamily \footnotesize},
  breaklines=true,
  morekeywords={true,false,warning,xlim,ylim},
  keywordstyle=\color{blue},
  stringstyle=\color{matlablilas},
  commentstyle={\color{matlabgreen} \itshape},
  numberstyle={\ttfamily \tiny},
  frame=leftline,
  showstringspaces=false,
  numbers=left,
  upquote=true,
}
\maketitle
\section{Single-server queues}
\subsection*{a)}
The first queueing system is simulated by initializing a variable that
keeps the queue length to zero and, at each slot,
\begin{itemize}
\item the queue length decreases by one if it's not empty,
  \item one arrival or two arrivals are generated with the same
    probability $a$, then the queue length is increased and the delay
    that the new packets will suffer are stored (they are known since
    the service time is deterministic),
  \item the queue length at the end of the slot is stored
\end{itemize}
What happens in the first slot is not stored in the output data
because the queue is in a transient state until the maximum service
time has passed.

Since the server always processes one packet per slot and the average
number of arrivals per slot is $\lambda = 3a$, the load factor of the
system is $\rho = \frac{\mu}{\lambda} = 3a$. The values of $a$ for
which the system will be simulated ($ a \in [0, \frac{1}{3}]$)
correspond to stable values of $\rho$, except for $a=\frac{1}{3}$.

If this system is simulated for 51 values of $a$ letting it run for
$10^5$ time slots, the average delays suffered by each packet are the
ones shown in Fig.~\ref{plot:queue1_delay} along with their 95\%
confidence intervals.
\begin{figure}[htbp]
  \centering
    \includegraphics[width=0.6\textwidth]{matlab/queue_1_delay}
    \caption{Average delay of the first queueing system}
    \label{plot:queue1_delay}
\end{figure}
As it is expected, when $a$ grows toward $1/3$ the average delay becomes bigger
and bigger because the system is going toward instability.

In
Fig.s~\ref{plot:queue1_sizes_stable},~\ref{plot:queue1_sizes_limit}~and~\ref{plot:queue1_sizes_unstable}
there are three realizations of the queue length of a system that ran
for 10000 slots. When $a = 1/4$ the system is stable so it can be seen
(Fig.~\ref{plot:queue1_sizes_stable}) that the queue length oscillates
but remains low and returns often to zero. When $a = 1/3$ the system's
load factor is $\rho = 1$ and the queue size tends to grow as time
goes on (Fig.~\ref{plot:queue1_sizes_limit}) because the queueing
system is unstable. In the last case ($a = 1/2$) there is always at
least one arrival per slot so the queue length can never decrease
(Fig.~\ref{plot:queue1_sizes_unstable}).
\begin{figure}[htbp]
  \centering
  \begin{subfigure}{0.5\textwidth}
    \centering
    \includegraphics[width=\textwidth]{matlab/queue1_sizes_1}
    \caption{$ a = 1/4 $}
    \label{plot:queue1_sizes_stable}
  \end{subfigure}%
  \begin{subfigure}{0.5\textwidth}
    \centering
    \includegraphics[width=\textwidth]{matlab/queue1_sizes_2}
    \caption{$ a = 1/3 $}
    \label{plot:queue1_sizes_limit}
  \end{subfigure}
  \begin{subfigure}{0.5\textwidth}
    \centering
    \includegraphics[width=\textwidth]{matlab/queue1_sizes_3}
    \caption{$ a = 1/2 $}
    \label{plot:queue1_sizes_unstable}
  \end{subfigure}
  \caption{Realizations of the queue length of the first system during
    10000 slots}
\end{figure}
\subsection*{b)}
In this case the service time is not bounded but we can pick a value
that will be larger than the first service time with arbitrary probability:
\begin{align}
  \Prob{Y \leq k} \geq \alpha \\
  1-(1-b)^k \geq \alpha \\
  k \geq \frac{\ln(1-\alpha)}{\ln(1-b)}
\end{align}
\begin{figure}[htbp]
  \centering
    \includegraphics[width=0.6\textwidth]{matlab/queue_2_delay}
    \caption{Average delay of Queue 2}
    \label{plot:queue2_delay}
\end{figure}


\begin{figure}[htbp]
  \centering
  \begin{subfigure}{0.5\textwidth}
    \centering
    \includegraphics[width=\textwidth]{matlab/queue2_sizes_1}
    \caption{Realization of the queue size}
    \label{plot:queue2_sizes_unstable}
  \end{subfigure}%
  \begin{subfigure}{0.5\textwidth}
    \centering
    \includegraphics[width=\textwidth]{matlab/queue2_sizes_2}
    \caption{Realization of the queue size}
    \label{plot:queue2_sizes_limit}
  \end{subfigure}
  \begin{subfigure}{0.5\textwidth}
    \centering
    \includegraphics[width=\textwidth]{matlab/queue2_sizes_3}
    \caption{Realization of the queue size}
    \label{plot:queue2_sizes_stable}
  \end{subfigure}
\end{figure}


\section{Capture probabilities}
\section{GeRaF}
\end{document}
