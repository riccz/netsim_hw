\documentclass[a4paper,oneside]{article}
\usepackage[T1]{fontenc}
\usepackage[utf8]{inputenc}
\usepackage[english]{babel}

\usepackage[margin=2.54cm]{geometry}
\usepackage{amsmath}
\usepackage{bbm}
\usepackage{siunitx}
\usepackage{listings}
\usepackage{color}
\usepackage{textcomp}
\usepackage{graphicx}
\usepackage{subcaption}
%\usepackage{changepage}
\usepackage[section]{placeins}
\usepackage{hyperref}

\definecolor{matlabgreen}{RGB}{28,172,0}
\definecolor{matlablilas}{RGB}{170,55,241}

\newcommand{\includecode}[1]{\lstinputlisting[caption={\ttfamily #1.m},label={lst:#1}]{matlab/#1.m}}
\newcommand{\inlinecode}[1]{\lstinline[basicstyle=\ttfamily,keywordstyle={},stringstyle={},commentstyle={\itshape}]{#1}}

\renewcommand{\vec}[1]{\underline{#1}}
\renewcommand{\Re}[1]{\operatorname{Re}\left[#1\right]}
\newcommand{\E}[1]{\operatorname{E}\left[#1\right]}
\newcommand{\norm}[1]{\left\lVert#1\right\rVert}
\newcommand{\abs}[1]{\left|#1\right|}
\newcommand{\F}[1]{\operatorname{\mathcal{F}}\left[#1\right]}
\newcommand{\ceil}[1]{\left\lceil#1\right\rceil}
\newcommand{\floor}[1]{\left\lfloor#1\right\rfloor}
\newcommand{\Prob}[1]{\operatorname{P}\left[#1\right]}
\newcommand{\ProbC}[2]{\operatorname{P}\left[#1\middle|#2\right]}
\newcommand{\ind}[1]{\operatorname{\mathbbm{1}}\left\{#1\right\}}
\newcommand{\distr}[0]{\sim}
\newcommand{\unif}[1]{\mathcal{U}_{#1}}

\newcolumntype{M}{>{$}r<{$}}

\author{Riccardo Zanol}
\title{Homework 3}

\begin{document}
\lstset{
  language=Matlab,
  basicstyle={\ttfamily \footnotesize},
  breaklines=true,
  morekeywords={true,false,warning,xlim,ylim},
  keywordstyle=\color{blue},
  stringstyle=\color{matlablilas},
  commentstyle={\color{matlabgreen} \itshape},
  numberstyle={\ttfamily \tiny},
  frame=leftline,
  showstringspaces=false,
  numbers=left,
  upquote=true,
}
\maketitle
\section{Single-server queues}
\subsection*{a)}
The first queueing system is simulated by initializing a queue length
to zero, and at each slot, decrese by one if not empty and simulate
one or two arrivals extractng a uniform and see when it's less than
$a$ oe $2a$. In this way the arrivals must wait for the next slot even
if the server is idle. At each arrival the delay that the job will
suffer is stored, since it depends only on the size of the
queue. Since the server always processes one job per slot ($\mu = 1$)
and the average number of arrivals per slot is $\lambda = 3a$, the
load factor of the system is $\rho = \frac{\mu}{\lambda} = 3a$. The
simulation skips the maximum service time (1 slot) to measure the
statistics only in stationary regime.

If we simulate this system for 51 values of $a \in [0, 1/3]$, letting
it run for $10^5$ time slots we get the average delays shown in
Fig.~\ref{plot:queue1_delay} along with their 95\% confidence intervals.
\begin{figure}[htbp]
  \centering
    \includegraphics[width=0.6\textwidth]{matlab/queue_1_delay}
    \caption{Average delay of Queue 1}
    \label{plot:queue1_delay}
\end{figure}
We see that as $a$ grows toward $1/3$ the average delay becomes bigger
and bigger because the system is going toward instability.

In
Fig.s~\ref{plot:queue1_sizes_stable},~\ref{plot:queue1_sizes_limit}~and~\ref{plot:queue1_sizes_unstable}
there are the results of the simulation of a queue length realization
plotted vs the time slots. As we can see when $\rho < 1$
(Fig.~\ref{plot:queue1_sizes_stable}) the queue length oscillates but
remains low and returns often to zero. At $\rho = 1$
(Fig.~\ref{plot:queue1_sizes_limit}) the system becomes unstable and
the queue starts to grow indefinitely, even if not very quickly. In
the last case the queue length cannot ever decrease so it becomes very
large after only a few slots (Fig.~\ref{plot:queue1_sizes_unstable}).
\begin{figure}[htbp]
  \centering
  \begin{subfigure}{0.5\textwidth}
    \centering
    \includegraphics[width=\textwidth]{matlab/queue1_sizes_1}
    \caption{Realization of the queue size}
    \label{plot:queue1_sizes_stable}
  \end{subfigure}%
  \begin{subfigure}{0.5\textwidth}
    \centering
    \includegraphics[width=\textwidth]{matlab/queue1_sizes_2}
    \caption{Realization of the queue size}
    \label{plot:queue1_sizes_limit}
  \end{subfigure}
  \begin{subfigure}{0.5\textwidth}
    \centering
    \includegraphics[width=\textwidth]{matlab/queue1_sizes_3}
    \caption{Realization of the queue size}
    \label{plot:queue1_sizes_unstable}
  \end{subfigure}
\end{figure}

\subsection*{b)}
In this case the service time is not bounded but we can pick a value
that will be larger than the first service time with arbitrary probability:
\begin{align}
  \Prob{Y \leq k} \geq \alpha \\
  1-(1-b)^k \geq \alpha \\
  k \geq \frac{\ln(1-\alpha)}{\ln(1-b)}
\end{align}
\begin{figure}[htbp]
  \centering
    \includegraphics[width=0.6\textwidth]{matlab/queue_2_delay}
    \caption{Average delay of Queue 2}
    \label{plot:queue2_delay}
\end{figure}


\begin{figure}[htbp]
  \centering
  \begin{subfigure}{0.5\textwidth}
    \centering
    \includegraphics[width=\textwidth]{matlab/queue2_sizes_1}
    \caption{Realization of the queue size}
    \label{plot:queue2_sizes_unstable}
  \end{subfigure}%
  \begin{subfigure}{0.5\textwidth}
    \centering
    \includegraphics[width=\textwidth]{matlab/queue2_sizes_2}
    \caption{Realization of the queue size}
    \label{plot:queue2_sizes_limit}
  \end{subfigure}
  \begin{subfigure}{0.5\textwidth}
    \centering
    \includegraphics[width=\textwidth]{matlab/queue2_sizes_3}
    \caption{Realization of the queue size}
    \label{plot:queue2_sizes_stable}
  \end{subfigure}
\end{figure}


\section{Capture probabilities}
\section{GeRaF}
\end{document}
