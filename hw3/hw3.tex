\documentclass[a4paper,oneside]{article}
\usepackage[T1]{fontenc}
\usepackage[utf8]{inputenc}
\usepackage[english]{babel}

\usepackage[margin=2.54cm]{geometry}
\usepackage{amsmath}
\usepackage{bbm}
\usepackage{siunitx}
\usepackage{listings}
\usepackage{color}
\usepackage{textcomp}
\usepackage{graphicx}
\usepackage{subcaption}
%\usepackage{changepage}
\usepackage[section]{placeins}
\usepackage{hyperref}

\definecolor{matlabgreen}{RGB}{28,172,0}
\definecolor{matlablilas}{RGB}{170,55,241}

\newcommand{\includecode}[1]{\lstinputlisting[caption={\ttfamily #1.m},label={lst:#1}]{matlab/#1.m}}
\newcommand{\inlinecode}[1]{\lstinline[basicstyle=\ttfamily,keywordstyle={},stringstyle={},commentstyle={\itshape}]{#1}}

\renewcommand{\vec}[1]{\underline{#1}}
\renewcommand{\Re}[1]{\operatorname{Re}\left[#1\right]}
\renewcommand{\approx}[0]{\simeq}
\newcommand{\E}[1]{\operatorname{E}\left[#1\right]}
\newcommand{\norm}[1]{\left\lVert#1\right\rVert}
\newcommand{\abs}[1]{\left|#1\right|}
\newcommand{\F}[1]{\operatorname{\mathcal{F}}\left[#1\right]}
\newcommand{\ceil}[1]{\left\lceil#1\right\rceil}
\newcommand{\floor}[1]{\left\lfloor#1\right\rfloor}
\newcommand{\Prob}[1]{\operatorname{P}\left[#1\right]}
\newcommand{\ProbC}[2]{\operatorname{P}\left[#1\middle|#2\right]}
\newcommand{\ind}[1]{\operatorname{\mathbbm{1}}\left\{#1\right\}}
\newcommand{\distr}[0]{\sim}
\newcommand{\unif}[1]{\mathcal{U}_{#1}}

\newcolumntype{M}{>{$}r<{$}}

\author{Riccardo Zanol}
\title{Homework 3}

\begin{document}
\lstset{
  language=Matlab,
  basicstyle={\ttfamily \footnotesize},
  breaklines=true,
  morekeywords={true,false,warning,xlim,ylim},
  keywordstyle=\color{blue},
  stringstyle=\color{matlablilas},
  commentstyle={\color{matlabgreen} \itshape},
  numberstyle={\ttfamily \tiny},
  frame=leftline,
  showstringspaces=false,
  numbers=left,
  upquote=true,
}
\maketitle
\section{Single-server queues}
\subsection*{a)}
The first queueing system is simulated by initializing to zero a
variable that represents the queue length and, at each slot,
\begin{itemize}
\item the queue length decreases by one if it's not empty,
  \item one arrival or two arrivals are generated with the same
    probability $a$, then the queue length is increased and the delay
    that the new packets will suffer are stored (they are known since
    the service time is deterministic),
  \item the queue length at the end of the slot is stored
\end{itemize}
What happens in the first slot is not considered in the output because
the queue is in a transient state until the maximum service time has
passed.

Since the server always processes one packet per slot and the average
number of arrivals per slot is $\lambda = 3a$, the load factor of the
system is $\rho = \frac{\mu}{\lambda} = 3a$. The values of $a$ for
which the system will be simulated ($ a \in [0, \frac{1}{3}]$)
correspond to stable values of $\rho$, except for $a=\frac{1}{3}$.

If this system is simulated for 51 values of $a$ and for $10^5$ time
slots, the average delay suffered by each packet is the one plotted in
Fig.~\ref{plot:queue1_delay} along with its 95\% confidence
interval.
\begin{figure}[htbp]
  \centering
    \includegraphics[width=0.6\textwidth]{matlab/queue_1_delay}
    \caption{Average delay of the first queueing system}
    \label{plot:queue1_delay}
\end{figure}
As expected, when $a$ approaches $1/3$ the average delay becomes
bigger and bigger because the system is going toward instability.

In
Fig.s~\ref{plot:queue1_sizes_stable},~\ref{plot:queue1_sizes_limit}~and~\ref{plot:queue1_sizes_unstable}
there are three realizations of the queue length for the same system
that ran for 10000 slots. When $a = 1/4$ the system is stable so it
can be seen (Fig.~\ref{plot:queue1_sizes_stable}) that the queue
length oscillates but remains low and returns often to zero. When $a =
1/3$ the system's load factor is $\rho = 1$ and the queue size tends
to grow as time goes on (Fig.~\ref{plot:queue1_sizes_limit}) because
the queueing system is unstable. In the last case ($a = 1/2$) there is
always at least one arrival per slot so the queue length can never
decrease (Fig.~\ref{plot:queue1_sizes_unstable}).
\begin{figure}[htbp]
  \centering
  \begin{subfigure}{0.5\textwidth}
    \centering
    \includegraphics[width=\textwidth]{matlab/queue1_sizes_1}
    \caption{$ a = 1/4 $}
    \label{plot:queue1_sizes_stable}
  \end{subfigure}%
  \begin{subfigure}{0.5\textwidth}
    \centering
    \includegraphics[width=\textwidth]{matlab/queue1_sizes_2}
    \caption{$ a = 1/3 $}
    \label{plot:queue1_sizes_limit}
  \end{subfigure}
  \begin{subfigure}{0.5\textwidth}
    \centering
    \includegraphics[width=\textwidth]{matlab/queue1_sizes_3}
    \caption{$ a = 1/2 $}
    \label{plot:queue1_sizes_unstable}
  \end{subfigure}
  \caption{Realizations of the queue length of the first system during
    10000 slots}
\end{figure}

\subsection*{b)}
The second queueing system is simulated like the first one with some
differences:
\begin{itemize}
  \item since the service time is random the delay that a packet will
    suffer is not known at the arrival time, so the arrival times of
    the packets are stored in a queue and are used at the end of the
    service time to compute the delay,
  \item the service time is not bounded anymore so, in order to set a
    transient duration after which the delay and queue size are
    recorded, a certain value $k$ that the service time does not
    exceed with probability $\alpha = 0.999$ is chosen. From
    \begin{align}
      \Prob{Y \leq k} & \geq \alpha \\
      1-(1-b)^k & \geq \alpha \\
      k & \geq \frac{\ln(1-\alpha)}{\ln(1-b)}
    \end{align}
    the value that is chosen is
    \begin{equation}
      k = \ceil{\frac{\ln10^{-3}}{\ln(1-b)}} .
    \end{equation}
\end{itemize}

For this system the arrival rate is $\lambda = 1/2$, the service rate
is $\mu = b$ so the load factor is $\rho = \frac{1}{2b}$. The values
of $b \in [1/2, 1]$ for which the system will be simulated correspond
to a stable $\rho$, except for $b=1/2$.

If this system is simulated for 51 values of $b$, letting it run for
$5 \cdot 10^5$ time slots, the average delays suffered by each packet
are the ones shown in Fig.~\ref{plot:queue2_delay} along with their
95\% confidence intervals.
\begin{figure}[htbp]
  \centering
    \includegraphics[width=0.6\textwidth]{matlab/queue_2_delay}
    \caption{Average delay of the second queueing system}
    \label{plot:queue2_delay}
\end{figure}
Like in the previous case as $\rho$ gets closer and closer to $1$, and
the corresponding $b$ decreases toward $1/2$, the average delay
increases and goes to infinity when $\rho = 1$ ($b = 1/2$).

The queue length realization for three simulations of this system with
different values of $b$ are shown in
Fig.s~\ref{plot:queue2_sizes_stable},~\ref{plot:queue2_sizes_limit}~and~\ref{plot:queue2_sizes_unstable},
where it can be seen that when the queueing system is unstable ($\rho
\geq 1$) the queue length tends to increase, like in the previous
case. When $\rho < 1$, instead, the queue length oscillates but it
remains close to zero.
\begin{figure}[htbp]
  \centering
  \begin{subfigure}{0.5\textwidth}
    \centering
    \includegraphics[width=\textwidth]{matlab/queue2_sizes_1}
    \caption{$ b = 1/3 $}
    \label{plot:queue2_sizes_unstable}
  \end{subfigure}%
  \begin{subfigure}{0.5\textwidth}
    \centering
    \includegraphics[width=\textwidth]{matlab/queue2_sizes_2}
    \caption{$ b = 1/2 $}
    \label{plot:queue2_sizes_limit}
  \end{subfigure}
  \begin{subfigure}{0.5\textwidth}
    \centering
    \includegraphics[width=\textwidth]{matlab/queue2_sizes_3}
    \caption{$ b = 2/3 $}
    \label{plot:queue2_sizes_stable}
  \end{subfigure}
  \caption{Realizations of the queue length of the second system
    during 10000 slots}
\end{figure}

\subsection*{c)}
For the two cases where the queueing system is unstable, i.e. $b =
1/3$ and $b=1/2$, there is no finite size of the queue that guarantees
$P_o = 10^{-5}$ because it grows indefinitely.

When $b = 2/3$, instead, the system is stable so the minimum queue
size that guarantees $P_o = 10^{-5}$ can be found doing a binary
search on the possible values of the queue size between two extremes.
Looking at Fig.~\ref{plot:queue2_sizes_stable}, an upper bound of $n_u
= 32$ looks big enough, while for the lower bound the minimum possible
value $n_l = 1$ is chosen. Then at each iteration of the binary
search:
\begin{itemize}
\item pick the integer $n$ in the middle of the interval $[n_l, n_u]$,
\item run a simulation to determine the overflow probability $P_o(n)$
  corresponding to the middle point,
\item if $P_o(n) < 10^{-5}$ take $[n_l, n]$ as the interval for the next iteration,
\item if $P_o(n) > 10^{-5}$ take $[n, n_u]$ as the interval for the next iteration,
\item if $P_o(n) = 10^{-5}$ or $n_u - n_l < 2$ stop the search and return $n$.
\end{itemize}
The overflow probability is determined by running the same simulation
of part b) with a finite queue length, counting the slots when the
queue is full and dividing by the number of simulated slots. The
number of simulated slots is quite high ($60 \cdot 10^5$) because with
a shorter interval there are too few overflows.

The result of the binary search is $n=16$ with $P_o(16) \approx
1.6\cdot10^{-5}$, so to have $P_o \leq 10^{-5}$ the queue size must be
increased by one: $n^* = n+1$.

At the end another simulation is ran with queue size $n^*$ that also
computes the 95\% confidence interval $[L, U]$ using the Gaussian
approximation of Theorem 2.4
\begin{align*}
  L(z) & \approx \frac{z}{n} - \frac{\eta}{n}\sqrt{z\left(1-\frac{z}{n}\right)} \\
  U(z) & \approx \frac{z}{n} + \frac{\eta}{n}\sqrt{z\left(1-\frac{z}{n}\right)}
\end{align*}
where $\eta = 1.96$, $z$ is the number of slots when the queue is full
and $n$ is the number of simulated slots. This last simulation also
checks that the number of overflows allows this approximation:
\begin{align*}
  z & \geq 6 \\
  n-z & \geq 6
\end{align*}

The results of the last simulation are:
\begin{align*}
  & P_o(n^* = 17) \approx 1.05 \cdot 10^{-5} \\
  & \text{with 95\% confidence interval} \quad [0.79 \cdot 10^{-5}, 1.31 \cdot 10^{-5}] .
\end{align*}
\section{Capture probabilities}
\section{GeRaF}
The GeRaF protocol is simulated assuming, like in the paper, that the
active nodes seen at each hop are independent and that the packet is
always forwarded to the best possible node.
%
The positions of the active nodes are generated before each hop by
drawing a Poisson distributed random variable with mean $M$, which is
the number of active nodes in the coverage area, and distributing the
nodes uniformly in a unit circle around the source node.
%
Then the node with minimum distance from the destination is selected
as the next source and this process is repeated until the destination
is in the coverage range.
%
When there are no active nodes or they are all farther than the source
from the destination, one hop is counted anyway and the forwarding is
tried again at the next iteration.

In
Fig.s~\ref{plot:geraf_sim_5},~\ref{plot:geraf_sim_10}~and~\ref{plot:geraf_sim_20}
there are the simulation results for three values of the distance $D$
between the source and the destination, as the average number of
neighbors $M$ varies. As the node density increases the average number
of hops goes toward the optimal value $\floor{D+1}$ because it is more
likely to find an active node in the optimal position: at the border
of the coverage area and aligned between the source and destination.
%
The results are the average and standard deviation of 1000
simulations.
\begin{figure}[htbp]
  \centering
  \begin{subfigure}{0.5\textwidth}
    \centering
    \includegraphics[width=\textwidth]{matlab/sim_mc_5}
    \caption{$ D = 5 $}
    \label{plot:geraf_sim_5}
  \end{subfigure}%
  \begin{subfigure}{0.5\textwidth}
    \centering
    \includegraphics[width=\textwidth]{matlab/sim_mc_10}
    \caption{$ D = 10 $}
    \label{plot:geraf_sim_10}
  \end{subfigure}
  \begin{subfigure}{0.5\textwidth}
    \centering
    \includegraphics[width=\textwidth]{matlab/sim_mc_20}
    \caption{$ D = 20 $}
    \label{plot:geraf_sim_20}
  \end{subfigure}
  \caption{Average number of hops to reach a destination at distance
    $D$ with $M$ average active nodes in the coverage area. The bars
    represent the standard deviation of the number of hops}
\end{figure}

The paper gives a recursive algorithm to compute an upper and lower
bound on the average number of hops: first it quantizes the unit
distance in $\nu$ intervals (so the whole distance contains $D\nu$
intervals) and computes the probabilities $\omega(i,k)$ of forwarding
from a source at distance $\gamma = i/\nu$ to a node in interval $i -
\nu + k$ (in the distance interval $(\frac{i - \nu + k - 1}{\nu},
\frac{i - \nu + k}{\nu}]$) as
  \begin{equation}
    \omega(i, k) = \begin{cases}
      e^{-MA(\frac{i - \nu + k - 1}{\nu}, \frac{i}{\nu})/\pi}
      - e^{-MA(\frac{i - \nu + k}{\nu}, \frac{i}{\nu})/\pi}
      \quad & k = 1, \dots \nu \\
      0 & \text{otherwise}
    \end{cases}
    \label{eq:omega}
  \end{equation}
  where $A(r, D)$ is the area of the intersection between the unit
  circle centered on the source and the circle of radius $r$, centered
  on the destination at distance $D$.
  
The area in equation~(\ref{eq:omega}) is given by the integral
\begin{equation}
  A(r, D) = 2\int_{D-1}^r a \arccos\left( \frac{a^2 + D^2 - 1}{2aD} \right) da
  \qquad D \geq 1, D-1 \leq r \leq D  
\end{equation}
  which has a closed solution in the paper. Using the solution given
  in the paper, however, yielded results that were different from the
  approximation of the integral computed using Monte Carlo and the
  Gauss quadrature rule. So in the recursive bound $A(r, D)$ is
  computed using the GQR with $n=64$ coefficients.

  The probabilities of equation~(\ref{eq:omega}), together with the
  probability of not forwarding the packet
  \begin{equation}
    \omega_0(i) = e^{-MA(\frac{i}{nu},\frac{i}{nu})/\pi}
  \end{equation}
  can be used to write two recursive relationships for the number of
  hops from position $i$ to the destination, assuming that the relay
  node is always placed, respectively, in the worst or the best point
  of its interval:
  \begin{align}
    n_1(i) &= \frac{
      1 + \sum_{k=1}^{\nu-1} \omega(i,k)n_1(i-\nu+k)
    }{
      1 - \omega_0(i) - \omega(i,\nu)
    }
    \label{eq:n1} \\
    n_2(i) &= \frac{
      1 + \sum_{k=1}^{\nu} \omega(i,k)n_2(i-\nu+k-1)
    }{
      1 - \omega_0(i)
    }
    \label{eq:n2}
    .
  \end{align}

  The bounds of equations~(\ref{eq:n1})~and~(\ref{eq:n2}) can be
  computed iteratively starting from the initial conditions
  \begin{equation}
    n_1(i) = n_2(i) = 1 \qquad i = 1, \dots, \nu 
  \end{equation}
  and the results for the same cases of the previous plots are shown
  in
  Fig.s~\ref{plot:geraf_rec_5},~\ref{plot:geraf_rec_10}~and~\ref{plot:geraf_rec_20}. The
  bounds were computed with $\nu = 50$ and it can be seen that, since
  they are very close, they give a precise value for the average
  number of hops under the stated assumptions.
\begin{figure}[htbp]
  \centering
  \begin{subfigure}{0.5\textwidth}
    \centering
    \includegraphics[width=\textwidth]{matlab/sim_rec_5}
    \caption{$ D = 5 $}
    \label{plot:geraf_rec_5}
  \end{subfigure}%
  \begin{subfigure}{0.5\textwidth}
    \centering
    \includegraphics[width=\textwidth]{matlab/sim_rec_10}
    \caption{$ D = 10 $}
    \label{plot:geraf_rec_10}
  \end{subfigure}
  \begin{subfigure}{0.5\textwidth}
    \centering
    \includegraphics[width=\textwidth]{matlab/sim_rec_20}
    \caption{$ D = 20 $}
    \label{plot:geraf_rec_20}
  \end{subfigure}
  \caption{Bounds on the average number of hops to reach a destination
    at distance $D$ with $M$ average active nodes in the coverage
    area, computed with the recursive approach}
\end{figure}


\end{document}
