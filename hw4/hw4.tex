\documentclass[a4paper,oneside]{article}
\usepackage[T1]{fontenc}
\usepackage[utf8]{inputenc}
\usepackage[english]{babel}

\usepackage[margin=2.54cm]{geometry}
\usepackage{amsmath}
\usepackage{bbm}
\usepackage{siunitx}
\usepackage{listings}
\usepackage{color}
\usepackage{textcomp}
\usepackage{graphicx}
\usepackage{subcaption}
%\usepackage{changepage}
\usepackage[section]{placeins}
\usepackage{hyperref}

\definecolor{matlabgreen}{RGB}{28,172,0}
\definecolor{matlablilas}{RGB}{170,55,241}

\newcommand{\includecode}[1]{\lstinputlisting[caption={\ttfamily #1.m},label={lst:#1}]{matlab/#1.m}}
\newcommand{\textinlinecode}[1]{\lstinline[basicstyle=\ttfamily,keywordstyle={},stringstyle={},commentstyle={\itshape}]{#1}}
\newcommand{\inlinecode}[1]{\ifmmode\text{\textinlinecode{#1}}\else\textinlinecode{#1}\fi}

\renewcommand{\vec}[1]{\underline{#1}}
\renewcommand{\Re}[1]{\operatorname{Re}\left[#1\right]}
\newcommand{\E}[1]{\operatorname{E}\left[#1\right]}
\newcommand{\norm}[1]{\left\lVert#1\right\rVert}
\newcommand{\abs}[1]{\left|#1\right|}
\newcommand{\F}[1]{\operatorname{\mathcal{F}}\left[#1\right]}
\newcommand{\ceil}[1]{\left\lceil#1\right\rceil}
\newcommand{\floor}[1]{\left\lfloor#1\right\rfloor}
\newcommand{\Prob}[1]{\operatorname{P}\left[#1\right]}
\newcommand{\ProbC}[2]{\operatorname{P}\left[#1\middle|#2\right]}
\newcommand{\ind}[1]{\operatorname{\mathbbm{1}}\left\{#1\right\}}
\newcommand{\distr}[0]{\sim}
\newcommand{\unif}[1]{\mathcal{U}_{#1}}

\newcolumntype{M}{>{$}r<{$}}

\author{Riccardo Zanol}
\title{Homework 4}

\begin{document}
\maketitle
\section{Discrete-event queue}
\lstset{
  language=Matlab,
  basicstyle={\ttfamily \footnotesize},
  breaklines=true,
  morekeywords={true,false,warning,xlim,ylim},
  keywordstyle=\color{blue},
  stringstyle=\color{matlablilas},
  commentstyle={\color{matlabgreen} \itshape},
  numberstyle={\ttfamily \tiny},
  frame=leftline,
  showstringspaces=false,
  numbers=left,
  upquote=true,
}

The discrete-event queue simulation is implemented as described in
Law's book with the difference that, instead of counting just the
cumulative statistics, the simulation stores the queue size and the
delays after each event has occurred. This is necessary to plot the
realization of the queue size and to compute the confidence intervals
for the average delay. The simulation does not store the busy/idle
state of the server because it is not used in the plots, but if it
keeps track of it to decide when to schedule the next departure.

The main function of the simulation,
\inlinecode{simulate_queue_times}, takes as input two functions to
generate the inter-arrival and service times, so they can be
continuous or discrete with any distribution.

The simulation is run with the same parameters of the previous
homework:
\begin{itemize}
\item $ b \in [1/2, 1]$,
\item slot duration $T_{slot} = \SI{1}{\s}$,
\item stop the simulation at $T_{sim} = 5 \cdot 10^5 T_{slot}$,
\item inter-arrival time $\tau = G(1/2)T_{slot}$,
\item service time $Y = G(b)T_{slot}$
\end{itemize}
where $G(p)$ is a geometrically distributed random variable with
success probability $p$ and minimum value 1. So the results are very
similar to the ones of homework 3: in Fig.~\ref{plot:queue_delay}
there is the average delay between the arrival of a packet and its
departure, along with its 95\% confidence interval.
\begin{figure}[htbp]
  \centering
    \includegraphics[width=0.6\textwidth]{queue/queue_delay}
    \caption{Average delay of the queueing system}
    \label{plot:queue_delay}
\end{figure}
In
Fig.s~\ref{plot:queue_sizes_unstable},~\ref{plot:queue_sizes_limit}~and~\ref{plot:queue_sizes_stable}
there are three realizations of the queue length for a simulation that
went on for $10000T_{slot}$ and for three values of $b$ that make the
system, respectively, unstable ($\rho > 1$ and $\rho = 1$) and stable
($\rho < 1$).
\begin{figure}[htbp]
  \centering
  \begin{subfigure}{0.5\textwidth}
    \centering
    \includegraphics[width=\textwidth]{queue/queue_sizes_1}
    \caption{$ b = 1/3 $}
    \label{plot:queue_sizes_unstable}
  \end{subfigure}%
  \begin{subfigure}{0.5\textwidth}
    \centering
    \includegraphics[width=\textwidth]{queue/queue_sizes_2}
    \caption{$ b = 1/2 $}
    \label{plot:queue_sizes_limit}
  \end{subfigure}
  \begin{subfigure}{0.5\textwidth}
    \centering
    \includegraphics[width=\textwidth]{queue/queue_sizes_3}
    \caption{$ b = 2/3 $}
    \label{plot:queue_sizes_stable}
  \end{subfigure}
  \caption{Realizations of the queue length of the system during 10000 slots}
\end{figure}

\section{UW polling}
\lstset{
  language=Python,
  basicstyle={\ttfamily \footnotesize},
  breaklines=true,
  morekeywords={},
  keywordstyle=\color{blue},
  stringstyle=\color{matlablilas},
  commentstyle={\color{matlabgreen} \itshape},
  numberstyle={\ttfamily \tiny},
  frame=leftline,
  showstringspaces=false,
  numbers=left,
  upquote=true,
}

In order to compute the metrics from the log files, the Python script
\inlinecode{polling.py} reads each file line by line and extracts the
information needed by matching the lines to some regular expressions.

To compute the packet delivery ratio at the MAC level the script
counts the packets that were received by the AUV, the packets that
were transmitted by the nodes and takes their ratio.  The lines in the
log file of the AUV that match the regular expression
\inlinecode{.*RX_DATA_ID_\\d+_FROM_NODE_(\\d+)} are grouped by node
id and counted. The same thing is done to the log files of the nodes
using the regular expression
\inlinecode{.*NODE\\((\\d+)\\)::TX_DATA_.*}.  The resulting PDR for
the two experiments are shown in Tab.~\ref{tab:pdr_mac}.
\begin{table}[h]
  \centering
  \begin{subtable}{0.5\textwidth}
    \centering
    \begin{tabular}{lr}
      \multicolumn{2}{c}{Experiment 35} \\
      \hline
      \multicolumn{1}{c}{Node} & \multicolumn{1}{c}{PDR} \\
      1 & 1.000000 \\
      2 & 1.000000 \\
      3 & 0.888889 \\
      4 & 1.000000 \\
      Whole network & 0.970588
    \end{tabular}
  \end{subtable}%
  \begin{subtable}{0.5\textwidth}
    \centering
    \begin{tabular}{lr}
      \multicolumn{2}{c}{Experiment 40} \\
      \hline
      \multicolumn{1}{c}{Node} & \multicolumn{1}{c}{PDR} \\
      1 & 1.000000 \\
      2 & 1.000000 \\
      3 & 0.888889 \\
      4 & 0.833333 \\
      Whole network & 0.943820
    \end{tabular}
  \end{subtable}
  \caption{Packet delivery ratio at the MAC level}
  \label{tab:pdr_mac}
\end{table}

To compute the packet delivery ratio at the application level the
scripts counts the packets in the same way as the previous paragraph,
with the difference that it now matches the packets that are received
from the upper layer with the expression
\inlinecode{.*NODE\\((\\d+)\\)::RECV_FROM_U_LAYERS_.*}. The computed
PDR is show in Tab.~\ref{tab:pdr_app} where it can be seen that in the
first experiment the network is working near its capacity, since the
application PDR is slightly less than the MAC PDR, while in the second
experiment more than half of the generated packets are still queued at
the end of the experiment. Looking at the way in which the PDRs are
sorted it can also be seen that the AUV prioritizes the nodes with a
lower id when it sends out the POLL packets.
\begin{table}[h]
  \centering
  \begin{subtable}{0.5\textwidth}
    \centering
    \begin{tabular}{lr}
      \multicolumn{2}{c}{Experiment 35} \\
      \hline
      \multicolumn{1}{c}{Node} & \multicolumn{1}{c}{PDR} \\
      1 & 1.000000 \\
      2 & 0.944444 \\
      3 & 0.888889 \\
      4 & 0.833333 \\
      Whole network & 0.916667 
    \end{tabular}
  \end{subtable}%
  \begin{subtable}{0.5\textwidth}
    \centering
    \begin{tabular}{lr}
      \multicolumn{2}{c}{Experiment 40} \\
      \hline
      \multicolumn{1}{c}{Node} & \multicolumn{1}{c}{PDR} \\
      1 & 0.734694 \\
      2 & 0.346939 \\
      3 & 0.326531 \\
      4 & 0.306122 \\
      Whole network & 0.428571 
    \end{tabular}
  \end{subtable}
  \caption{Packet delivery ratio at the application level}
  \label{tab:pdr_app}
\end{table}

To compute the throughput of each node and of the network the script
just has to divide the count of the received packets used in the first
paragraph by the last timestamp of the AUV node. The results are shown
in Tab.~\ref{tab:thr}.
\begin{table}[h]
  \centering
  \begin{subtable}{0.5\textwidth}
    \centering
    \begin{tabular}{lr}
      \multicolumn{2}{c}{Experiment 35} \\
      \hline
      \multicolumn{1}{c}{Node} & \multicolumn{1}{c}{Throughput} \\
      1 & 0.711191 pkts/min \\
      2 & 0.671680 pkts/min \\
      3 & 0.632170 pkts/min \\
      4 & 0.592659 pkts/min \\
      Whole network & 2.607699 pkts/min
    \end{tabular}
  \end{subtable}%
  \begin{subtable}{0.5\textwidth}
    \centering
    \begin{tabular}{lr}
      \multicolumn{2}{c}{Experiment 40} \\
      \hline
      \multicolumn{1}{c}{Node} & \multicolumn{1}{c}{Throughput} \\
      1 & 1.427335 pkts/min \\
      2 & 0.674019 pkts/min \\
      3 & 0.634371 pkts/min \\
      4 & 0.594723 pkts/min \\
      Whole network & 3.330448 pkts/min
    \end{tabular}
  \end{subtable}
  \caption{Throughput}
  \label{tab:thr}
\end{table}

For the computation of the overhead the number of every kind of packet
must be extracted from the log files. According to the protocol rules
the AUV sends TRIGGER and POLL packets, while the other nodes send
PROBE and DATA packets. The size of the control information in each
kind of packet is shown in Tab.~\ref{tab:pktsizes}.
\begin{table}[h]
  \centering
  \begin{tabular}{lr}
    \multicolumn{1}{c}{Packet} & \multicolumn{1}{c}{Size} \\
    \hline
    TRIGGER & 48 bits \\
    PROBE & 72 bits \\
    POLL & 24 bits \\
    DATA header & 61 bits
  \end{tabular}
  \caption{Size of the control information contained by each kind of packet}
  \label{tab:pktsizes}
\end{table}
Multiplying the number of packets of each kind sent by the AUV and the
nodes by the size of their control information, the script obtains the
total number of control bits sent. The overhead is calculated as the
number of control bits divided by the number of received data packets
and is shown in Tab.~\ref{tab:overh}.
\begin{table}[h]
  \centering
  \begin{tabular}{lr}
    \multicolumn{1}{c}{Experiment} & \multicolumn{1}{c}{Overhead} \\
    \hline
    35 &  190.848485 bits/pkt \\
    40 & 112.059524 bits/pkt
  \end{tabular}
  \caption{Overhead}
  \label{tab:overh}
\end{table}
By looking at the log files it is possible to see that the lower
overhead in the second case is due to the fact that the PROBEs sent by
the nodes ask to be allowed to transmit many packets, while in the previous
case most of the time the nodes only have one packet ready to be sent.

To compute the packet delays the script needs to record the time when
a specific packet was sent and received so it extracts from the AUV
logs the information needed using the regular expression
\inlinecode{(?P<time>[\\d\\.]+)Uwpolling_.*RX_DATA_ID_(?P<pkt_id>\\d+)_FROM_NODE_(?P<node_id>\\d+)}
and stores the packet ids and the associated times grouped by node id.
From the log files of the other nodes the information about the times
when the packets are received from the application layer is extracted
using the expression
\inlinecode{(?P<time>[\\d\\.]+)Uwpolling_NODE\\((?P<node_id>\\d+)\\)::RECV_FROM_U_LAYERS_ID_(?P<pkt_id>\\d+)}. Then
for each one of the received packets, a corresponding packet id is
looked up in the data relative to the sending node.

Since the nodes are not synchronized the difference between the
timestamps must take into account the clock shift between the sending
node and the AUV. In Tab.~\ref{tab:clockshifts} there are the clock
shifts between the nodes during the two experiments.
\begin{table}[h]
  \centering
  \begin{subtable}{0.5\textwidth}
    \centering
    \begin{tabular}{lr}
      \multicolumn{2}{c}{Experiment 35} \\
      \hline
      \multicolumn{1}{c}{Node} & \multicolumn{1}{c}{Clock shift} \\
      1 & 3 s \\
      2 & 3 s \\
      3 & 3 s \\
      4 & 3 s \\
      AUV & 0 s
    \end{tabular}
    \caption{Referenced to the AUV}
  \end{subtable}%
  \begin{subtable}{0.5\textwidth}
    \centering
    \begin{tabular}{lr}
      \multicolumn{2}{c}{Experiment 40} \\
      \hline
      \multicolumn{1}{c}{Node} & \multicolumn{1}{c}{Clock shift} \\
      1 & 0 s \\
      2 & 2 s \\
      3 & 5 s \\
      4 & 8 s \\
      AUV & 11 s
    \end{tabular}
    \caption{Referenced to node 1}
  \end{subtable}
  \caption{Clock shifts}
  \label{tab:clockshifts}
\end{table}
So the delay suffered by a packet sent by node $i$ at time $t_s$ and
received by the AUV at time $t_r$ is $ t_r - t_s - \Delta t_i $ where
the clock shift between the node and the AUV is
\begin{equation*}
  \Delta t_i = \begin{cases}
    3 & \text{experiment 35} \\
    -11 + \overline{\Delta t}_i & \text{experiment 40}
  \end{cases}
\end{equation*}
using for $\overline{\Delta t}_i$ the values of
Tab.~\ref{tab:clockshifts}.

The average delays are shown in Tab.~\ref{tab:delay}.
\begin{table}[h]
  \centering
  \begin{subtable}{0.5\textwidth}
    \centering
    \begin{tabular}{lr}
      \multicolumn{2}{c}{Experiment 35} \\
      \hline
      \multicolumn{1}{c}{Node} & \multicolumn{1}{c}{Avg. delay} \\
      1 & 75.777278 s \\
      2 & 70.546059 s \\
      3 & 97.251375 s \\
      4 & 102.188267 s \\
      Whole network & 85.638182 s
    \end{tabular}
  \end{subtable}%
  \begin{subtable}{0.5\textwidth}
    \centering
    \begin{tabular}{lr}
      \multicolumn{2}{c}{Experiment 40} \\
      \hline
      \multicolumn{1}{c}{Node} & \multicolumn{1}{c}{Avg. delay} \\
      1 & 470.549583 s \\
      2 & 133.377088 s \\
      3 & 133.191906 s \\
      4 & 538.686233 s \\
      Whole network & 350.220946 s
    \end{tabular}
  \end{subtable}
  \caption{Average packet delays}
  \label{tab:delay}
\end{table}
\end{document}
