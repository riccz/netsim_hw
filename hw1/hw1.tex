\documentclass{article}

\usepackage[T1]{fontenc}
\usepackage[utf8]{inputenc}
\usepackage[english]{babel}

\usepackage{amsmath}
\usepackage{siunitx}
\usepackage{hyperref}
\usepackage{listings}
\usepackage{color}
\usepackage{textcomp}
\usepackage{graphicx}

\definecolor{matlabgreen}{RGB}{28,172,0}
\definecolor{matlablilas}{RGB}{170,55,241}

\newcommand{\includecode}[1]{\lstinputlisting[caption={\ttfamily #1.m},
    label={lst:#1}]{matlab/#1.m}}
\newcommand{\inlinecode}[1]{\lstinline[basicstyle=\ttfamily,keywordstyle={}]{#1}}

\author{Riccardo Zanol}
\title{Homework 1}

\begin{document}
\lstset{
  language=Matlab,
  basicstyle={\ttfamily \footnotesize},
  breaklines=true,
  morekeywords={true,false,warning,xlim,ylim},
  keywordstyle=\color{blue},
  stringstyle=\color{matlablilas},
  commentstyle={\color{matlabgreen} \itshape},
  numberstyle={\ttfamily \tiny},
  frame=leftline,
  showstringspaces=false,
  numbers=left,
  upquote=true,
}
\maketitle
\begin{enumerate}
\item
\item The confidence intervals for the mean of the 48 uniform random
  variables generated 1000 times are shown in
  Fig.~\ref{uniform_mean_ci}. $51$ of them don't contain the true
  value of the mean (0.5) but this is expected as the probability that
  this intervals contain 0.5 is $\approx 5\%$.
  \begin{figure}[h]
    \centering
    \includegraphics[width=0.7\textwidth]{matlab/uniform_mean_ci}
    \caption{Uniform mean CIs}
    \label{uniform_mean_ci}
  \end{figure}
  The plot shape is in accordance with the central limit theorem that
  allows the distribution of the sample mean to be approximated with a
  gaussian.
\item
\item The mean for the $n$ uniform random variables generated in the
  script \inlinecode{ex4.m} for $ 1 \leq n \leq 100 $ are plotted in
  Fig.~\ref{uniform_mean}. The variances with their confidence
  intervals at level $\gamma = 95\%$ computed with the bootstrap
  method are plotted in Fig.~\ref{uniform_var}. It can be seen that
  the mean goes toward the true value $0.5$ as $1/n$ while the
  variance goes toward $1/12$ as $1/\sqrt{n}$ (and also the CI
  decreases with the same speed).
  \begin{figure}[h]
    \centering
    \includegraphics[width=0.7\textwidth]{matlab/uniform_mean_n}
    \caption{Mean for $n$ uniform random variables}
    \label{uniform_mean}
  \end{figure}
  \begin{figure}[h]
    \centering
    \includegraphics[width=0.7\textwidth]{matlab/uniform_variance_n}
    \caption{Variance for $n$ uniform random variables}
    \label{uniform_var}
  \end{figure}
  The prediction interval (Fig.~\ref{pred_int_unif}) does not depend
  on $n$, so as soon as there are enough samples ($n=39$) it remains
  around the same value. The theoretical values for a 95\% prediction
  interval are $[0.025, 0.975]$.
  \begin{figure}[h]
    \centering
    \includegraphics[width=0.7\textwidth]{matlab/uniform_prediction_interval_n}
    \caption{Prediction interval for $n$ uniform random variables}
    \label{pred_int_unif}
  \end{figure}
\item In the same way of point 2 the plot of the mean CIs is computed
  (fig.~\ref{gaussian_mean_ci}), it has $\approx 5\%$ of the CIs that
  don't contain the true value 0 and the same gaussian shape.
  \begin{figure}[h]
    \centering
    \includegraphics[width=0.7\textwidth]{matlab/gaussian_mean_ci}
    \caption{Mean CIs for the gaussian r.v.s}
    \label{gaussian_mean_ci}
  \end{figure}

  As in point 4 the mean (Fig.~\ref{gaussian_mean_n}), variance with
  CIs (Fig.\ref{gaussian_var_n}) and the prediction interval
  (Fig.\ref{gaussian_pred_int_n}) are computed for $1 \leq n \leq 100$.
  \begin{figure}[h]
    \centering
    \includegraphics[width=0.7\textwidth]{matlab/normal_mean_n}
    \caption{Mean for $n$ gaussian random variables}
    \label{gaussian_mean_n}
  \end{figure}
  \begin{figure}[h]
    \centering
    \includegraphics[width=0.7\textwidth]{matlab/normal_variance_n}
    \caption{Variance for $n$ gaussian random variables}
    \label{gaussian_var_n}
  \end{figure}
    \begin{figure}[h]
    \centering
    \includegraphics[width=0.7\textwidth]{matlab/normal_prediction_interval_n}
    \caption{Prediction interval for $n$ gaussian random variables}
    \label{gaussian_pred_int_n}
  \end{figure}

\end{enumerate}
\end{document}
